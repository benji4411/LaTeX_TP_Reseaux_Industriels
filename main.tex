\documentclass{rapportECC}
\usepackage{lipsum}
\title{Rapport UBS - Reseaux Industriels} %Titre du fichier
\usepackage{lipsum} 
\usepackage{biblatex} %Imports biblatex package
\addbibresource{main.bib} %Import the bibliography file
\usepackage{appendix} % Package pour gérer les annexes
\usepackage{float}
\begin{document}

%----------- Informations du rapport ---------

\titre{Réseaux Industriels} %Titre du fichier .pdf

\sujet{\LaTeX Approfondi} %Nom du sujet

\Encadrants{Julien \textsc{Delorme} } %Nom de l'enseignant

\eleves{Alexandre \textsc{Perez} \\
		Benjamin \textsc{Mahoudeau} \\
        Leonardo \textsc{Montoya Obeso}} %Nom des élèves

%----------- Initialisation -------------------
        
\fairemarges %Afficher les marges
\fairepagedegarde %Créer la page de garde
\tabledematieres %Créer la table de matières

%------------ Corps du rapport ----------------


\section{Introduction} 

Le but de ce TP est de mettre en pratique nos connaissances théoriques du bus CAN par l'utilisation d'une maquette automobile pédagogique EXXOTEST (voir annexe \ref{fig:maquette EXXOTEST}). Cette maquette est composée d'équipements réels tels que des feux xénon, des clignotants, des essuie-glaces, des rétroviseurs et des vitres électriques. De plus, la maquette dispose d'une partie simulée pour le moteur avec des potentiomètres (voir annexe \ref{fig:potentiomètres}) permettant de changer les rapports, d'augmenter le régime moteur, de définir le niveau d'essence, etc...

% ========================================================================================================
% 1 Prise en main de la maquette
% ========================================================================================================

\section{TP1: Prise en main de la maquette}

% --------------------------------------------------------------------------------------------------------
% 1.1 Bus CAN de la maquette
% --------------------------------------------------------------------------------------------------------

\subsection{Bus CAN de la maquette}

\subsubsection*{Identifier le nombre de bus CAN présents sur la laquette ainsi que leur vitesse.}

Comme indiqué en haut à gauche du schéma de câblage de la maquette (voir annexe \ref{fig:schema cablage}), il y a en tout 4 bus CAN: le bus CAN DIAG, le bus CAN I/S, le bus CAN CONF, ainsi que le bus CAN CAR. Les CAN CONF et CAR sont tous deux des CAN LS (low speed) de 125 kbit/s, alors que le CAN I/S est un CAN HS (high speed) avec une vitesse de 500 kbit/s.

\subsubsection*{Quelles fonctions remplissent chacun de ces bus sur la maquette.}

Les fonctions remplies par chacun des bus CAN de la maquette sont les suivantes :

\begin{itemize}
    \item Le bus CAN DIAG, pour diagnostic, permet de récupérer des informations importantes sur les équipements afin de les diagnostiquer.
    \item Le bus CAN I/S, pour inter-système, s'occupe des liaisons importantes de commandes et de sécurité.
    \item Le bus CAN CONF, pour confort, s'occupe de la liaison des équipements d'indication et d'aide à la conduite.
    \item Le bus CAN CAR, pour carrosserie, s'occupe de la liaison des informations moteurs et des équipements extérieurs.
\end{itemize}

\subsubsection*{Décrire la structure d'un paquet CAN.}

Comme décrit sur la figure \ref{fig:structure trame CAN}, chaque trame est séparée par 3 bits au minimum. Une trame CAN est découpée en 7 morceaux, ayant chacun une fonction différente :

\begin{itemize}
    \item Un champ de Début de 1 bit marque le début de la trame.
    \item Un champ ID de 11 bits sert à identifier l'émetteur de l'information. Avec 11 bits, il est possible de coder 2047 IDs différents, c'est donc le maximum d'équipements qui peuvent être connectés à un bus CAN.
    \item Un champ Commande contient le DLC (data length code) de 4 bits qui spécifie la taille du champ d'information.
    \item Le champ information contient les données de la trame et peut avoir une taille entre 0 et 8 octets.
    \item Un champ CRC de 15 bits est calculé à partir de l'ensemble des champs transmis jusqu'alors et permet la détection d'erreur.
    \item Un champ ACK de 2 bits permet l'acquittement d'un message.
    \item Un champ EOF de 7 bits indique la fin de la trame.
\end{itemize}

Comme on peut le constater, dans une trame CAN, le champ d'information est seulement de 64 bits au maximum, ce qui est vraiment peu pour une trame de 107 bits au total. Cependant, l'intérêt principal de l'utilisation du protocole CAN réside dans sa robustesse et sa résilience aux pannes et aux bruits, en plus d'une quantité de câblage réduite. C'est pourquoi le bus CAN est largement utilisé dans le domaine automobile.

\insererfigure{./images/structure_trames_CAN.png}{.7}{structure des trames CAN, source: \cite{exxotest}}{structure trame CAN}

\subsubsection*{Donner le nombre d'ID différents présents sur chaque bus identifiés (exports de donnéés depuis le logiciel MuxTrace).}

Pour les 3 bus étudiés (I/S, CONF, et CAR), le nombre d'ID différents présents sur chaque bus est respectivement de 22, 70 et 23.

% --------------------------------------------------------------------------------------------------------
% 1.2 Port OBDII
% --------------------------------------------------------------------------------------------------------

\subsection{Port OBDII}

\subsubsection*{A quoi sert le port OBDII ?}

Comme décrit sur le site \cite{klavkarr}, la norme OBDII a été instaurée à l'origine par la CARB (Californian Air Resources Board) afin de surveiller les émissions polluantes des véhicules. La norme requiert que le véhicule surveille de manière continue le bon fonctionnement du moteur tout au long de sa durée de vie.

Le protocole OBD permet le diagnostic embarqué d'un véhicule. C'est un système électronique responsable de l'auto-diagnostic et de la production de rapports destinés aux professionnels de la réparation. Grâce à ce système, les techniciens ont accès à des informations sur les sous-systèmes du véhicule, permettant ainsi le suivi de sa performance et l'analyse des besoins en réparations.

Il existe plusieurs normes de l'OBD : 
\begin{itemize}
    \item L'OBD ou OBDI standardise le connecteur pour assurer son uniformité sur tous les véhicules. En revanche, le protocole de communication demeure plus ou moins spécifique en fonction des marques.
    \item L'OBDII est arrivé aux États-Unis en 1996 pour spécifier des protocoles communs. En plus de cela, contrairement à l'OBDI qui est connecté à l’extérieur de la console d'une voiture, l'OBDII est intégré au véhicule.
    \item L'EOBD, pour European OBD, reprenant l'OBDII, est spécifique pour les véhicules européens.
\end{itemize}

    
\subsubsection*{Reccherche bibliographique sur le web : câblage de connecteur, monde d'échange entre les outils et le véhicle.}

Comme on peut le voir à l'annexe \ref{fig:Port OBDII}, le port OBDII est composé de 16 broches, chacune ayant un rôle spécifique. Parmi elles, il y a :

\begin{itemize}
    \item 2 broches pour le positif et le négatif du J1850 (SAE) utilisant le protocole de modulation de durée d’impulsion (VPW).
    \item 2 broches pour récupérer la masse du châssis et la masse du signal.
    \item 2 broches pour la ligne K et la ligne L utilisant un protocole de communication série asynchrone.
    \item 2 broches pour le LOW et le HIGH du protocole CAN.
    \item 7 broches non standard réservées au constructeur.
    \item 1 broche d'alimentation connectée à la batterie du véhicule pour alimenter les outils de scan. Elle est soit de type A avec une tension de 12V pour les voitures, ou de type B avec une tension de 24V pour les poids lourds.
\end{itemize}

Les informations sont générées par les unités de contrôle du moteur puis sont analysées par les outils connectés au port OBDII présent dans le véhicule.

% ========================================================================================================
% 2 Utilisation du logiciel MuxTrace
% ========================================================================================================

\section{TP1: Utilisation du logiciel MuxTrace}

% --------------------------------------------------------------------------------------------------------
% 2.3 Etude practique : régime moteur
% --------------------------------------------------------------------------------------------------------

\subsection{Etude pratique : régime moteur}

\subsubsection*{Observer sur le bus IS la trame 208.}

\dots

\subsubsection*{Faites varier le régime moteur et donnez vos observations sur le champs de données. Par exemple, vous pouvez faire varier la vitesse par pas de 1000tr.min\textsuperscript{-1}.}

\dots

\subsubsection*{Pouves-vous en déduire une suite logique sur la variation du régime moteur ? Dans tous les cas, donnez vos arguments.}

\dots

% --------------------------------------------------------------------------------------------------------
% 2.2 Etude practique : vitesse véhicule
% --------------------------------------------------------------------------------------------------------

\subsection{Etude pratique : vitesse véhicule}

\subsubsection*{Sur le bus IS, identifiez la ou les trames; qui évolent en foction de la vitesse du véhiule.}

\dots

\subsubsection*{Observez la trme 0x44D. Quelle est sa taille et sa fréquence ?}

\dots

\subsubsection*{Faites varier la vitesse du véhicle et donnez vos observations sur les champs de données; Par exemple , vous pouvez fair varier la viteese par pas de 10 de 0 à 100Km/h.}

\dots

\subsubsection*{Pouvez-vous en déduire une suite loguique ur la variationd e la vitesse véhicule . Dans tous les cas; donnez vos arguments.}

% --------------------------------------------------------------------------------------------------------
% 2.3 Etude practique : Comodo de phares
% --------------------------------------------------------------------------------------------------------
\subsection{Etude pratique : Comodo de phares}

\subsubsection*{Actionnez le comodo de phares, et observez l'activité sur les 3 bis CA? pnservanmes depuis MuxTrace.}

\dots

\subsubsection*{Quels identificateurs de trames ont un rapport avec les commandes d'éclariage, de signalisation et d'essuyage ?}

\dots

\subsubsection*{Sur quels bus avez-vous observé ces identificateurs ?}

\dots

\subsubsection*{Quelle est la taille de la trame concernée ?}

\dots

\subsubsection*{Actionnez la commande d'éclariage et établissez un tableau des fonctions comandées.}

\dots

% --------------------------------------------------------------------------------------------------------
% 2.4 Etude practique : Rétroviseurs
% --------------------------------------------------------------------------------------------------------

\subsection{Etude pratique : Rétroviseurs}

\subsubsection*{Faites fonctionner les rétroviseurs du côté droit et du côté gauche, et soyez attentif aux trames. Quel est l'identificateur de tarme qui correspond aux rétroviseurs gauche et droit ?  Sur quel bus ave-vous dait l'observation ?}

\subsubsection*{Quelle est la taille des données de cette trame ?}

\dots

\subsubsection*{A quoi sert le premier octet de donnée ?}

\dots

\subsubsection*{Observer le premier octet de donnnés et établisseaz un tableau des actions  réalissées en fonction de la valeur de l'octet ?}

\dots

\subsubsection*{Utiliser le générateur interactif afin de'émettre la trame de commande des rétroviseurs observée précédemment. Vérifiez les trames de commandes observées précédeññent en associant des touches de votre calvier pour piloter les rétroviseurs. Vous pouvez faire de même avec les leve-vitre ?}

\dots

% --------------------------------------------------------------------------------------------------------
% 2.5 Etude practique : Faites votre choix
% --------------------------------------------------------------------------------------------------------

\subsection{Etude pratique : Faites votre choix}

\subsubsection*{En fonction de votre avancement dans la séance de TP, vous pouvez exposer une ou deux trames supplémentaires de votre choix sur une ou pluseirus bus de la maquette. De la meme manière que précédemment, donnez la tailles de données, vos observations sous forme de tableau et justifiez vos analyses.}

\dots

% ========================================================================================================
% 3 Utilisation du logiciel MuxTrace et DBEdit
% ========================================================================================================

\section{TP2: Utilisation du logiciel MuxTrace et DBEdit}

\subsubsection*{Basé sur les observations faites dans la section précédente, créer une base de donnée pour chaque bus et chaque identifiant de paquets analysées.}

\dots

\subsubsection*{Lorsque les fichiers de base de données sont prets, chagez-les depuis votre projet MuxTrace. Vérifiez que vos observations sont cohérentes lorsque vous actionnez les différents commandes concernés de la maquette.}

\dots

\subsubsection*{Qu'en déduisezvous de l'intéret de ces bases de données ?}

\dots

\subsubsection*{Utilisez maintenant le générateur interactif de MuxTrace pour fafire varier les champs de données d'un ID de trame connu dans la base de données. Quel est l'intéret de ce lien avec la vase de donnée par rapport aux signaux d'une trame ?}

\dots


% ========================================================================================================
% 4 Utilisation du mode programmation sous MuxTrace
% ========================================================================================================

\section{TP3: Utilisation du mode programmation sous MuxTrace}

% --------------------------------------------------------------------------------------------------------
% 4.1 chargement d'un exemple
% --------------------------------------------------------------------------------------------------------

\subsubsection*{Etude practique : chargement d'un exemple}

% --------------------------------------------------------------------------------------------------------
% 4.2 Création de votre exemple
% --------------------------------------------------------------------------------------------------------

\subsubsection*{Etude practique : Création de votre exemple}

\section{Conclusion}

\newpage


\printbibliography

% Ajout de l'annexe
\newpage
\begin{appendices}
\section{Annexe : }
% Ajout d'une image
\subsection{Illustration complémentaire}
\insererfigure{./images/maquette.jpg}{.6}{Illustration de la maquette EXXOTEST}{maquette EXXOTEST}
\insererfigure{./images/potentiomètres_simu.jpg}{.8}{Illustration des potentiomètres et interrupteurs de la partie simulée}{potentiomètres}
\insererfigure{./images/schema_cablage.jpg}{.8}{Schéma du cablage de la maquette}{schema cablage}
\insererfigure{./images/OBDII_pins.png}{.7}{Illustration port OBDII}{Port OBDII}
\end{appendices}

\end{document}