\documentclass{rapportECC}
\usepackage{lipsum}
\title{Rapport UBS - Reseaux Industriels} %Titre du fichier
\usepackage{lipsum} 
\usepackage{biblatex} %Imports biblatex package
\addbibresource{main.bib} %Import the bibliography file
\usepackage{appendix} % Package pour gérer les annexes
\usepackage{float}
\usepackage{booktabs} % toprule, midrule, ...
\usepackage{makecell}
\usepackage{comment} % comment for some debugging text formats
\begin{document}

%----------- Informations du rapport ---------

\titre{Réseaux Industriels} %Titre du fichier .pdf

\sujet{\LaTeX Approfondi} %Nom du sujet

\Encadrants{Julien \textsc{Delorme} } %Nom de l'enseignant

\eleves{Alexandre \textsc{Perez} \\
		Benjamin \textsc{Mahoudeau} \\
        Leonardo \textsc{Montoya Obeso}} %Nom des élèves

%----------- Initialisation -------------------
        
\fairemarges %Afficher les marges
\fairepagedegarde %Créer la page de garde
\tabledematieres %Créer la table de matières

%------------ Corps du rapport ----------------


\section{Introduction} 

Ce document résulte des travaux de groupe menés durant les TP du cours de Réseaux Industriels du programme de Master SESI pour l'année académique 2023-2024 à l'Université Bretagne Sud.

Le but de ce TP est de mettre en pratique nos connaissances théoriques du bus CAN par l'utilisation d'une maquette automobile pédagogique EXXOTEST (voir annexe \ref{fig:maquette EXXOTEST}). Cette maquette est composée d'équipements réels tels que des feux xénon, des clignotants, des essuie-glaces, des rétroviseurs et des vitres électriques. De plus, la maquette dispose d'une partie simulée pour le moteur avec des potentiomètres (voir annexe \ref{fig:potentiomètres}) permettant de changer les rapports, d'augmenter le régime moteur, de définir le niveau d'essence, entre autres.

% ========================================================================================================
% 1 Prise en main de la maquette
% ========================================================================================================

\section{TP1: Prise en main de la maquette}

% --------------------------------------------------------------------------------------------------------
% 1.1 Bus CAN de la maquette
% --------------------------------------------------------------------------------------------------------

\subsection{Bus CAN de la maquette}

\subsubsection*{Identifier le nombre de bus CAN présents sur la laquette ainsi que leur vitesse.}

Comme indiqué en haut à gauche du schéma de câblage de la maquette (voir annexe \ref{fig:schema cablage}), il y a en tout 4 bus CAN: le bus CAN DIAG, le bus CAN I/S, le bus CAN CONF, ainsi que le bus CAN CAR. Les CAN CONF et CAR sont tous deux des CAN LS (low speed) de 125 kbit/s, alors que le CAN I/S est un CAN HS (high speed) avec une vitesse de 500 kbit/s.

\subsubsection*{Quelles fonctions remplissent chacun de ces bus sur la maquette.}

Les fonctions remplies par chacun des bus CAN de la maquette sont les suivantes :

\begin{itemize}
    \item Le bus CAN DIAG, pour diagnostic, permet de récupérer des informations importantes sur les équipements afin de les diagnostiquer.
    \item Le bus CAN I/S, pour inter-système, s'occupe des liaisons importantes de commandes et de sécurité.
    \item Le bus CAN CONF, pour confort, s'occupe de la liaison des équipements d'indication et d'aide à la conduite.
    \item Le bus CAN CAR, pour carrosserie, s'occupe de la liaison des informations moteurs et des équipements extérieurs.
\end{itemize}

\subsubsection*{Décrire la structure d'un paquet CAN.}

La Figure \ref{fig:structure trame CAN} montre que chaque trame est séparée par 3 bits au minimum. Une trame CAN est découpée en 7 morceaux, ayant chacun une fonction différente :

\begin{itemize}
    \item Un champ de Début de 1 bit marque le début de la trame.
    \item Un champ ID de 11 bits sert à identifier l'émetteur de l'information. Avec 11 bits, il est possible de coder 2047 IDs différents, c'est donc le maximum d'équipements qui peuvent être connectés à un bus CAN.
    \item Un champ Commande contient le DLC (data length code) de 4 bits qui spécifie la taille du champ d'information.
    \item Le champ information contient les données de la trame et peut avoir une taille entre 0 et 8 octets.
    \item Un champ CRC de 15 bits est calculé à partir de l'ensemble des champs transmis jusqu'alors et permet la détection d'erreur.
    \item Un champ ACK de 2 bits permet l'acquittement d'un message.
    \item Un champ EOF de 7 bits indique la fin de la trame.
\end{itemize}


Comme on peut le voir, le champ d'information d'une trame CAN a une longueur maximale de 64 bits, longueur courte pour une trame de 107 bits au total. Cependant, le principal avantage de l'utilisation du protocole CAN réside dans sa robustesse et sa résilience aux pannes et au bruit, outre un câblage réduit. C'est pourquoi le bus CAN est largement utilisé dans le domaine automobile

\insererfigure{./images/structure_trames_CAN.png}{.7}{structure des trames CAN, source: \cite{exxotest}}{structure trame CAN}

\subsubsection*{Donner le nombre d'ID différents présents sur chaque bus identifiés (exports de donnéés depuis le logiciel MuxTrace).}

Pour les 3 bus étudiés (I/S, CONF, et CAR), le nombre d'ID différents présents sur chaque bus est respectivement de 22, 70 et 23.

% --------------------------------------------------------------------------------------------------------
% 1.2 Port OBDII
% --------------------------------------------------------------------------------------------------------

\subsection{Port OBDII}

\subsubsection*{A quoi sert le port OBDII ?}

Comme décrit sur le site \cite{klavkarr}, la norme OBDII a été instaurée à l'origine par la CARB, ``Californian Air Resources Board'' pour son acronyme en anglais, afin de surveiller les émissions polluantes des véhicules. La norme requiert que le véhicule surveille de manière continue le bon fonctionnement du moteur tout au long de sa durée de vie.\pagebreak

Le protocole OBD permet le diagnostic embarqué d'un véhicule. C'est un système électronique responsable de l'auto-diagnostic et de la production de rapports destinés aux professionnels de la réparation. Grâce à ce système, les techniciens ont accès à des informations sur les sous-systèmes du véhicule, permettant ainsi le suivi de sa performance et l'analyse des besoins en réparations.

Certaines versions de la norme OBDII sont énumérées ci-dessous.
\begin{itemize}
    \item L'OBD ou OBDI standardise le connecteur pour assurer son uniformité sur tous les véhicules. En revanche, le protocole de communication demeure plus ou moins spécifique en fonction des marques.
    \item L'OBDII est arrivé aux États-Unis en 1996 pour spécifier des protocoles communs. En plus de cela, contrairement à l'OBDI qui est connecté à l'extérieur de la console d'une voiture, l'OBDII est intégré au véhicule.
    \item L'EOBD, pour European OBD, reprenant l'OBDII, est spécifique pour les véhicules européens.
\end{itemize}

    
\subsubsection*{Reccherche bibliographique sur le web : câblage de connecteur, mode d'échange entre les outils et le véhicle.}

Comme on peut le voir à l'annexe \ref{fig:Port OBDII}, le port OBDII est composé de 16 broches, chacune ayant un rôle spécifique. Parmi elles, il y a :

\begin{itemize}
    \item 2 broches pour le positif et le négatif du J1850 (SAE) utilisant le protocole de modulation de durée d’impulsion (VPW).
    \item 2 broches pour récupérer la masse du châssis et la masse du signal.
    \item 2 broches pour la ligne K et la ligne L utilisant un protocole de communication série asynchrone.
    \item 2 broches pour le LOW et le HIGH du protocole CAN.
    \item 7 broches non standard réservées au constructeur.
    \item 1 broche d'alimentation connectée à la batterie du véhicule pour alimenter les outils de scan. Elle est soit de type A avec une tension de 12V pour les voitures, ou de type B avec une tension de 24V pour les poids lourds.
\end{itemize}

Les informations sont générées par les unités de contrôle du moteur puis sont analysées par les outils connectés au port OBDII présent dans le véhicule.

% ========================================================================================================
% 2 Utilisation du logiciel MuxTrace
% ========================================================================================================

\section{TP1: Utilisation du logiciel MuxTrace}

Dans cette partie, nous allons utiliser le logiciel MuxTrace afin d'analyser les trames circulant dans les bus CAN. Pour cela, nous allons nous interfacer avec les bus I/S, CONF, et CAR grâce à la plaque de branchement de la maquette (voir illustration du branchement, annexe \ref{fig:Tableau interface}). Les signaux des bus CAN sont ensuite envoyés sur un boîtier USB-MUX-6C6L, avec le CAN CONF sur le port 1, le CAN I/S sur le port 2, et le CAN CAR sur le port 3 (voir illustration annexe \ref{fig:Boitier usb}). Cela permet, au final, de récupérer les informations des trames sur le logiciel MuxTrace. \pagebreak

Sur le logiciel MuxTrace, nous configurons chaque bus avec le bon débit en utilisant la détection automatique, comme illustré dans la figure \ref{fig:Detection auto débit}.

\insererfigure{./images/detection_auto_débit.PNG}{.5}{Outil de détection automatique du débit sur le logiciel MuxTrace}{Detection auto débit}

Nous pouvons ensuite observer les trames pour les 3 bus CAN sur le logiciel MuxTrace, comme on peut le voir dans la figure \ref{fig:Liste trames bus}.

\insererfigure{./images/liste_trames_bus.PNG}{1}{Observation des trames sur le logiciel MuxTrace}{Liste trames bus}

% --------------------------------------------------------------------------------------------------------
% 2.3 Etude practique : régime moteur
% --------------------------------------------------------------------------------------------------------

\subsection{Etude pratique : régime moteur}

\subsubsection*{Observer sur le bus IS la trame 208.}

Sur MuxTrace nous pouvons voir que la trame d'ID 208 du bus IS contient des données de 8 octets et emets avec une période de 10ms.

\subsubsection*{Faites varier le régime moteur et donnez vos observations sur le champs de données. Par exemple, vous pouvez faire varier la vitesse par pas de 1000tr.min\textsuperscript{-1}.}

Lorsque nous faisons varier le régime moteur, nous constatons que le premier octet de la trame 208 varie. Nous avons reporté dans le tableau \ref{tab:regime moteur trame 208} la valeur du premier octet obtenue en faisant varier la vitesse par pas de 1000tr.min\textsuperscript{-1} de 0tr.min\textsuperscript{-1} à 5000tr.min\textsuperscript{-1}.

\begin{comment}
\begin{table}[H]
    \begin{tabular}{|l|l|l|}
    \hline
    \textbf{Régime moteur (tr.min\textsuperscript{-1})} & \textbf{Premier octet (hexa)} & \textbf{Premier octet (base 10)} \\ \hline
    0  & 00       & 00 \\
    1000                               & 21       & 33 \\
    2000                               & 3F       & 63 \\
    3000                               & 5E       & 94 \\
    4000                               & 7D       & 125\\   
    5000                               & 9B       & 155\\  
    \end{tabular}
    \caption{Premier octet de la trame 208 en fonction du régime moteur}
    \label{tab:regime moteur trame 208}
\end{table}
\end{comment}

\begingroup
\begin{table}[H]
    \centering
    \begin{tabular}{c c c}
        \toprule
        \makecell{\textbf{Régime moteur}\\ \textbf{(tr.min\textsuperscript{-1})}} & \makecell{\textbf{Premier octet}\\ \textbf{(hexa)}} & \makecell{\textbf{Premier octet}\\ \textbf{(base 10)}} \\
        \midrule
        0    & 00 & 00 \\
        1000 & 21 & 33 \\
        2000 & 3F & 63 \\
        3000 & 5E & 94 \\
        4000 & 7D & 125 \\
        5000 & 9B & 155 \\
        \bottomrule
    \end{tabular}
    \caption{Premier octet de la trame 208 en fonction du régime moteur}
    \label{tab:regime_moteur_trame_208}
\end{table}
\endgroup

\subsubsection*{Pouves-vous en déduire une suite logique sur la variation du régime moteur ? Dans tous les cas, donnez vos arguments.}

En observant les valeurs en base 10, on constate la suite suivante:

\begin{itemize}
    \item de 0 à 1000: +33
    \item de 1000 à 2000: +30
    \item de 2000 à 3000: +31
    \item de 3000 à 4000: +31
    \item de 4000 à 5000: +30
\end{itemize}

On a donc une suite logique d'environ 31 par pas de 1000tr.min\textsuperscript{-1}.

% --------------------------------------------------------------------------------------------------------
% 2.2 Etude practique : vitesse véhicule
% --------------------------------------------------------------------------------------------------------

\subsection{Etude pratique : vitesse véhicule}

\subsubsection*{Sur le bus IS, identifiez la ou les trames qui évolent en fonction de la vitesse du véhiule.}

Lorsque l'on modifie la vitesse, nous constatons un changement sur le premier, troisième et cinquième octet de la trame 44D ainsi que le premier octet de la trame 38D.

\subsubsection*{Observez la trame 0x44D. Quelle est sa taille et sa fréquence ?}

La trame 0x44D a une taille de 8 octets et une fréquence de 40ms

\subsubsection*{Faites varier la vitesse du véhicle et donnez vos observations sur les champs de données. Par exemple, vous pouvez fair varier la viteese par pas de 10 de 0 à 100Km/h.}

Lorsque nous faisons varier la vitesse du véhicule, nous constatons que le premier, troisième et cinquième octet de la trame 44D varient avec la même valeur. Nous avons reporté dans le tableau \ref{tab:vitesse véhicule trame 44D} la valeur obtenue en faisant varier la vitesse par pas de 10km/h de 0km/h à 100km/h.

\begin{comment}
    \begin{table}[H]
    \begin{tabular}{|l|l|l|}
    \hline
    \textbf{Vitesse véhicule (tr/min)} & \textbf{Octets 1,3 et 5 (hexa)} & \textbf{Otets 1,3 et 5  (base 10)} \\ \hline
    0  & 00       & 00 \\
    10 & 03       & 03 \\
    20 & 08       & 08 \\
    30 & 0B       & 11 \\
    40 & 0F       & 15\\   
    50 & 13       & 19\\
    60 & 17       & 23\\
    70 & 1B       & 27\\
    80 & 1E       & 30\\
    90 & 22       & 34\\  
    100& 26       & 38\\
    \end{tabular}
    \caption{Octets 1, 3 et 5 de la trame 44D en fonction de la vitesse du véhicule}
    \label{tab:vitesse véhicule trame 44D}
\end{table}
\end{comment}

\begingroup
\begin{table}[H]
    \centering
    \begin{tabular}{c c c}
    \toprule
    \makecell{\textbf{Vitesse véhicule}\\ \textbf{(km/h)}} & \makecell{\textbf{Octets 1,3 et 5}\\ \textbf{(hexa)}} & \makecell{\textbf{Octets 1,3 et 5}\\ \textbf{(base 10)}} \\
    \midrule
    0  & 00 & 00 \\
    10 & 03 & 03 \\
    20 & 08 & 08 \\
    30 & 0B & 11 \\
    40 & 0F & 15 \\   
    50 & 13 & 19 \\
    60 & 17 & 23 \\
    70 & 1B & 27 \\
    80 & 1E & 30 \\
    90 & 22 & 34 \\  
    100& 26 & 38 \\
    \bottomrule
    \end{tabular}
    \caption{Octets 1, 3 et 5 de la trame 44D en fonction de la vitesse du véhicule.}
    \label{tab:vitesse véhicule trame 44D}
\end{table}
\endgroup


\subsubsection*{Pouvez-vous en déduire une suite logique sur la variation de la vitesse du véhicule ? Dans tous les cas, donnez vos arguments.}

En observant les valeurs en base 10, on constate une suite logique de 3.8, soit environ 4 par pas de 10km/h.

% --------------------------------------------------------------------------------------------------------
% 2.3 Etude practique : Comodo de phares
% --------------------------------------------------------------------------------------------------------
\subsection{Etude pratique : Comodo de phares}

\subsubsection*{Actionnez le comodo de phares, et observez l'activité sur les 3 bus CAN observables depuis MuxTrace.}

\dots

\subsubsection*{Quels identificateurs de trames ont un rapport avec les commandes d'éclairage, de signalisation et d'essuyage ?}

\dots

\subsubsection*{Sur quels bus avez-vous observé ces identificateurs ?}

\dots

\subsubsection*{Quelle est la taille de la trame concernée ?}

\dots

\subsubsection*{Actionnez la commande d'éclariage et établissez un tableau des fonctions comandées.}

\dots

% --------------------------------------------------------------------------------------------------------
% 2.4 Etude practique : Rétroviseurs
% --------------------------------------------------------------------------------------------------------

\subsection{Etude pratique : Rétroviseurs}

\subsubsection*{Faites fonctionner les rétroviseurs du côté droit et du côté gauche, et soyez attentif aux trames. Quel est l'identificateur de tarme qui correspond aux rétroviseurs gauche et droit ?  Sur quel bus ave-vous dait l'observation ?}

\subsubsection*{Quelle est la taille des données de cette trame ?}

\dots

\subsubsection*{A quoi sert le premier octet de donnée ?}

\dots

\subsubsection*{Observer le premier octet de donnnés et établissez un tableau des actions  réalisées en fonction de la valeur de l'octet.}

\dots

\subsubsection*{Utiliser le générateur interactif afin d'émettre la trame de commande des rétroviseurs observée précédemment. Vérifiez les trames de commandes observées précédemment en associant des touches de votre clavier pour piloter les rétroviseurs. Vous pouvez faire de même avec les lève-vitres ?}

\dots

% --------------------------------------------------------------------------------------------------------
% 2.5 Etude practique : Faites votre choix
% --------------------------------------------------------------------------------------------------------

\subsection{Etude pratique : Faites votre choix}

\subsubsection*{En fonction de votre avancement dans la séance de TP, vous pouvez exposer une ou deux trames supplémentaires de votre choix sur une ou plusieurs bus de la maquette. De la même manière que précédemment, donnez la taille de données, vos observations sous forme de tableau et justifiez vos analyses.}

\dots

% ========================================================================================================
% 3 Utilisation du logiciel MuxTrace et DBEdit
% ========================================================================================================

\section{TP2: Utilisation du logiciel MuxTrace et DBEdit}

\subsubsection*{Basé sur les observations faites dans la section précédente, créer une base de donnée pour chaque bus et chaque identifiant de paquets analysées.}

\dots

\subsubsection*{Lorsque les fichiers de base de données sont prêts, chagez-les depuis votre projet MuxTrace. Vérifiez que vos observations sont cohérentes lorsque vous actionnez les différentes commandes concernées de la maquette.}

\dots

\subsubsection*{Qu'en déduisez-vous de l'intéret de ces bases de données ?}

\dots

\subsubsection*{Utilisez maintenant le générateur interactif de MuxTrace pour faire varier les champs de données d'un ID de trame connu dans la base de données. Quel est l'intéret de ce lien avec la base de donnée par rapport aux signaux d'une trame ?}

\dots


% ========================================================================================================
% 4 Utilisation du mode programmation sous MuxTrace
% ========================================================================================================

\section{TP3: Utilisation du mode programmation sous MuxTrace}

\textit{Sous MuxTrace, il est possible d'avoir un mode programmation qui permet d'automatiser la génération interactive de trames sur des actions clavier ou sur des évènements conditionnels liés à la configuration du véhicule.}

% --------------------------------------------------------------------------------------------------------
% 4.1 chargement d'un exemple
% --------------------------------------------------------------------------------------------------------

\subsubsection*{Etude practique : chargement d'un exemple}

\textit{Dans un premier temps, on vous demande de charger le fichier d'exemple de dll fournit en séances de TP dans l'outil MuxTrace.}

\textit{Analysez et observez cet example et donnez vos remarques sur le comportement et les interactions de cet exemple sur la maquete.}

% --------------------------------------------------------------------------------------------------------
% 4.2 Création de votre exemple
% --------------------------------------------------------------------------------------------------------

\subsubsection*{Etude practique : Création de votre exemple}

\textit{Proposez un scénario d'exemple que vous souhaitez réalisér et décrivez-le de manière détaillée (identifiants de paquets, évènements conditionnels, \dots). Ces précisions porteront une attention toute particulière pour évaluer le Compte-rendu de TP avec le code source et la dll qui sera remis en fin de séances de TPs.}

\textit{Développez votre programme en C en partant de l'exemple foruni dans le répertoire d'installation de MuxTrace afin de créers votre dll d'exmplequi réalisera le scénario que vous aurez inventé.}

\section{Conclusion}

\newpage


\printbibliography

% Ajout de l'annexe
\newpage
\begin{appendices}
\section{Annexe : }
% Ajout d'une image
\subsection{Illustration complémentaire}
\insererfigure{./images/maquette.jpg}{.7}{Illustration de la maquette EXXOTEST}{maquette EXXOTEST}
\insererfigure{./images/potentiomètres_simu.jpg}{.7}{Illustration des potentiomètres et interrupteurs de la partie simulée}{potentiomètres}
\insererfigure{./images/schema_cablage.jpg}{.7}{Schéma du cablage de la maquette}{schema cablage}
\insererfigure{./images/OBDII_pins.png}{.7}{Port OBDII}{Port OBDII}
\insererfigure{./images/tableau_interface.jpg}{.6}{Illustration des branchements sur la plaque de la maquette}{Tableau interface}
\insererfigure{./images/boitier_usb.jpg}{.6}{Illustration du boitier USB-MUX-6C6L}{Boitier usb}
\end{appendices}

\end{document}