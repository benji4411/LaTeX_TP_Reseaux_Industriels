\documentclass{rapportECC}
\usepackage{lipsum}
\title{Rapport UBS - Reseaux Industriels} %Titre du fichier
\usepackage{lipsum} 
\usepackage{biblatex} %Imports biblatex package
\addbibresource{bibtex.bib} %Import the bibliography file
\usepackage{appendix} % Package pour gérer les annexes
\usepackage{url}
\usepackage{float}
\begin{document}

%----------- Informations du rapport ---------

\titre{Réseaux Industriels} %Titre du fichier .pdf

\sujet{\LaTeX Approfondi} %Nom du sujet

\Encadrants{Julien \textsc{Delorme} } %Nom de l'enseignant

\eleves{Alexandre \textsc{Perez} \\
		Benjamin \textsc{Mahoudeau} } %Nom des élèves

%----------- Initialisation -------------------
        
\fairemarges %Afficher les marges
\fairepagedegarde %Créer la page de garde
\tabledematieres %Créer la table de matières

%------------ Corps du rapport ----------------


\section{Introduction} 

\subsection{Contexte et objectifs}

\subsection{Énoncé de la problématique}

\section{TP1: Prise en main de la maquette}

\subsection{Bus CAN de la maquette}

\subsection{Port OBDII}
Comme décris sur le site https://www.outilsobdfacile.fr/presentation-de-l-obd.php, La norme OBDII a été instaurée à l'origine par la CARB (Californian Air Resources Board) afin de surveiller les émissions polluantes des véhicules.  
\\
\\
L'OBD, requiert que le véhicule surveille de manière continue le bon fonctionnement du moteur tout au long de sa durée de vie.
\\
\\
Il existe plusieurs normes de l'OBD : 
\begin{itemize}
    \item L'OBD ou OBDI standardise le connecteur pour assurer son uniformité sur tous les véhicules. En revanche, le protocole de communication demeure plus ou moins spécifique en fonction des marques.
    \item L'OBDII \ref{fig: Port OBDII} est arrivé en 1996 aux Etats Unis pour spécifier des protocoles communs.
    \item L'EOBD, pour European OBD, reprenant l'OBDII est spécifique pour les véhicules européens.
\end{itemize}


\section{TP2: Utilisation du logiciel MuxTrace}

\subsection{Etude pratique : régime moteur}

\subsection{Etude pratique : vitesse véhicule}

\subsection{Etude pratique : Comodo de phares}

\subsection{Etude pratique : Rétroviseurs}

\subsection{Etude pratique : Faites votre choix}

\section{TP3: Utilisation du logiciel MuxTrace et DBEdit}

\section{TP4: Utilisation du mode programmation sous MuxTrace}

\subsection{Etude pratique : Chargement d'un exemple}

\subsection{Etude pratique :Création de votre exemple}

\section{Conclusion}

\newpage


\printbibliography

% Ajout de l'annexe
\newpage
\begin{appendices}
\section{Annexe : }
% Ajout d'une image
\subsection{Illustration complémentaire}
\insererfigure{logos/port_OBDII.png}{10cm}{Illustartion port OBDII}{Port OBDII}
\end{appendices}

\end{document}